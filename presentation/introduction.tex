\frame{
  \frametitle{Introduction}

  \begin{itemize}
    \item Billiard ball bouncing in a square
    \item Assume no gravity or friction
  \end{itemize}
}

\subsection{Basic Notation}

\frame{
  \frametitle{Basic Notation}
  \begin{definition}
    A table $T \subset \R^2$ is the unit square. Vertical sides are labelled with a $v$. Horizontal sides are labelled with an $h$.
  \end{definition}

  \begin{figure}
    \includegraphics[width=2in]{square_with_sides.png}
  \end{figure}
}

\subsection{Example}

\frame{
  \frametitle{Example}

  \begin{figure}
    \includegraphics[width=3in]{example/example_1.png}
  \end{figure}
  \begin{eqnarray}
    {\color{red}vv}
  \end{eqnarray}
}

\frame{
  \frametitle{Example}

  \begin{figure}
    \includegraphics[width=3in]{example/example_2.png}
  \end{figure}
  \begin{eqnarray}
    vv{\color{red}vhv}
  \end{eqnarray}
}
\frame{
  \frametitle{Example}

  \begin{figure}
    \includegraphics[width=3in]{example/example_3.png}
  \end{figure}
  \begin{eqnarray}
    vvvhv{\color{red}vvv}
  \end{eqnarray}
}

\frame{
  \frametitle{Example}

  \begin{figure}
    \includegraphics[width=3in]{example/example_4.png}
  \end{figure}
  \begin{eqnarray}
    vvvhvvvv{\color{red}vhv}
  \end{eqnarray}
}

\frame{
  \frametitle{Example}

  \begin{figure}
    \includegraphics[width=3in]{example/example_5.png}
  \end{figure}
  \begin{eqnarray}
    vvvhvvvvvhv{\color{red}vv}
  \end{eqnarray}
}

\frame{
  \frametitle{Example}

  \begin{figure}
    \includegraphics[width=3in]{example/example_6.png}
  \end{figure}
  \begin{eqnarray}
    vvvhvvvvvhvvv{\color{red}vhv}
  \end{eqnarray}
}

\frame{
  \frametitle{Resulting Sequence}
  \begin{eqnarray}
    vvvhvvvvvhvvvvhv
  \end{eqnarray}
}

\subsection{Outline}

\frame{
  \frametitle{Presentation Outline}
  \tableofcontents
}

\subsection{Problem Statement}

\frame{
  \frametitle{Problem Statement}

   Problem: Given a sequence of $v$ and $h$ collisions, determine if it is a valid collision sequence.
}

\subsection{Primary and Secondary Sides}

\frame{
  \frametitle{Secondary Side Theorem}

  \begin{theorem}
    At least one side will never have more than one consecutive occurrence in a valid collision string.
  \end{theorem}
}

\frame{
  \frametitle{Secondary Side Theorem Examples}

  \begin{example}
    Valid: $vhhhvhhhv$
  \end{example}

  \begin{example}
    Valid: $vhvhvhv$
  \end{example}

  \begin{example}
    Valid: $vvvvvhvvvvhvvvvvhvvvv$
  \end{example}

  \begin{example}
    Invalid: $vvhhhvvvhhhvvhhh$
  \end{example}

  \begin{example}
    Invalid: $vhhhvvhvh$
  \end{example}
}

\frame{
  \frametitle{Secondary Side Theorem Proof}

  \begin{itemize}
    \item A billiard ball trajectory must be a line in the tiled grid with slope $m$.
    \item Case 1: $m = 1$.
    \item Case 2: $m < 1$ or $m > 1$.
  \end{itemize}
}

\frame{
  \frametitle{Secondary Side Theorem Proof}

  If $m = 1$, $v$ and $h$ alternate.

  % TODO: show a picture here
}

\frame{
  \frametitle{Secondary Side Theorem Proof}

  If $m < 1$, there must exist an $h$ between each $v$.

  If $m > 1$, similar argument holds.

  % TODO: show a picture of a single square in the lattice, 
}

\frame{
  \frametitle{Notation}
  \begin{definition}
    \textbf{Secondary side}: a side which never has more than one consecutive occurrences.
    \textbf{Primary side}: a side which is not a secondary side.
  \end{definition}

  \pause

  \begin{definition}
    \textbf{Primary substring:} a subsequence from the collision string which contains a consecutive sequence of primary sides.
  \end{definition}

  \pause

  \begin{example}
    \textbf{Collision string}: $vvhvvvhvvhvvvh$
    \textbf{Secondary Side}: $h$
    \textbf{Primary Side}: $v$
    \textbf{Primary substrings}: $vv$, $vvv$
  \end{example}
}

