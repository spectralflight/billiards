%
%  beamer_template
%
%  Created by Christopher Schommer-Pries on 2012-02-16.
%  Copyright (c) 2012. All rights reserved.
%


\documentclass{beamer}		%% The Beamer document class formats for slides. 

\usepackage{presentation}

\title{Billiards}
\author{Jonathan Allen, John Wang}
\institute[MIT]{Massachusetts Institute of Technology}
\date{November $22^\text{nd}$, 2013}

\begin{document}

\graphicspath{ {figures/} }

\frame{

\titlepage

}

\section{Introduction}

\frame{
  \frametitle{Introduction}

  \begin{itemize}
    \item Billiard ball bouncing in a square
    \item Assume no gravity or friction
  \end{itemize}
}

\subsection{Basic Notation}

\frame{
  \frametitle{Basic Notation}
  \begin{definition}
    A table $T \subset \R^2$ is the unit square. Vertical sides are labelled with a $v$. Horizontal sides are labelled with an $h$.
  \end{definition}

  \begin{figure}
    \includegraphics[width=2in]{square_with_sides.png}
  \end{figure}
}

\subsection{Example}

\frame{
  \frametitle{Example}

  \begin{eqnarray*}
    {\color{red}vv}
  \end{eqnarray*}
  \begin{figure}
    \includegraphics[width=3in]{example/example_1.png}
  \end{figure}
}

\frame{
  \frametitle{Example}

  \begin{eqnarray*}
    vv{\color{red}vhv}
  \end{eqnarray*}
  \begin{figure}
    \includegraphics[width=3in]{example/example_2.png}
  \end{figure}
}
\frame{
  \frametitle{Example}

  \begin{eqnarray*}
    vvvhv{\color{red}vvv}
  \end{eqnarray*}
  \begin{figure}
    \includegraphics[width=3in]{example/example_3.png}
  \end{figure}
}

\frame{
  \frametitle{Example}

  \begin{eqnarray*}
    vvvhvvvv{\color{red}vhv}
  \end{eqnarray*}
  \begin{figure}
    \includegraphics[width=3in]{example/example_4.png}
  \end{figure}
}

\frame{
  \frametitle{Example}

  \begin{eqnarray*}
    vvvhvvvvvhv{\color{red}vv}
  \end{eqnarray*}
  \begin{figure}
    \includegraphics[width=3in]{example/example_5.png}
  \end{figure}
}

\frame{
  \frametitle{Example}

  \begin{eqnarray*}
    vvvhvvvvvhvvv{\color{red}vhv}
  \end{eqnarray*}
  \begin{figure}
    \includegraphics[width=3in]{example/example_6.png}
  \end{figure}
}

\frame{
  \frametitle{Resulting Sequence}
  \begin{eqnarray*}
    vvvhvvvvvhvvvvhv
  \end{eqnarray*}
}

\subsection{Outline}

\frame{
  \frametitle{Presentation Outline}
  \tableofcontents
}

\subsection{Problem Statement}

\frame{
  \frametitle{Problem Statement}

   Problem: Given a sequence of $v$ and $h$ collisions, determine if it is a valid collision sequence.
}

\frame{
  \frametitle{Basic Notation}

  \begin{definition}
    \emph{$v$ collision}: when the ball collides with a $v$ side
  \end{definition}

  \begin{definition}
    \emph{$h$ collision}: when the ball collides with an $h$ side
  \end{definition}

  \begin{definition}
    \emph{Collision sequence ($\alpha$)}: a sequence of $v$ and $h$ collisions which starts and ends with an $h$ collision.
  \end{definition}
}


\section{Lemmas}

\subsection{Tiling}

\frame{
  \frametitle{Representing Collision Strings}

  \begin{itemize}
    \item Reflect squares about each side to create a tiling
    \item Solutions become lines in the plane
    \item Intersections become places where collisions occur
  \end{itemize}
}

\frame{
  \frametitle{Representing Collision Strings}

  \begin{example}
    Tiling of $\vec{x}_0 = (0, 0.5)$ and $\vec{v} = (0.25, 0.25)$.
  \end{example}

  \begin{columns}
    \begin{column}{0.45\textwidth}
      \begin{figure}
        \includegraphics[width=2.1in]{abab.pdf}
      \end{figure}
    \end{column}
    \begin{column}{0.45\textwidth}
      \begin{figure}
        \includegraphics[width=1.7in]{tiling.png}
      \end{figure}
    \end{column}
  \end{columns}
}

\frame{
  \frametitle{Representing Collision Strings}

  \begin{example}
    Tiling of $\vec{x}_0 = (0, 0.5)$ and $\vec{v} = (0.25, 0.25)$.
  \end{example}

  \begin{columns}
    \begin{column}{0.45\textwidth}
      \begin{figure}
        \includegraphics[width=2.1in]{tiling_real_step_1.png}
      \end{figure}
    \end{column}
    \begin{column}{0.45\textwidth}
      \begin{figure}
        \includegraphics[width=1.7in]{tiling_tiled_step_1.png}
      \end{figure}
    \end{column}
  \end{columns}
}

\frame{
  \frametitle{Representing Collision Strings}

  \begin{example}
    Tiling of $\vec{x}_0 = (0, 0.5)$ and $\vec{v} = (0.25, 0.25)$.
  \end{example}

  \begin{columns}
    \begin{column}{0.45\textwidth}
      \begin{figure}
        \includegraphics[width=2.1in]{tiling_real_step_2.png}
      \end{figure}
    \end{column}
    \begin{column}{0.45\textwidth}
      \begin{figure}
        \includegraphics[width=1.7in]{tiling_tiled_step_2.png}
      \end{figure}
    \end{column}
  \end{columns}
}

\frame{
  \frametitle{Representing Collision Strings}

  \begin{example}
    Tiling of $\vec{x}_0 = (0, 0.5)$ and $\vec{v} = (0.25, 0.25)$.
  \end{example}

  \begin{columns}
    \begin{column}{0.45\textwidth}
      \begin{figure}
        \includegraphics[width=2.1in]{tiling_real_step_3.png}
      \end{figure}
    \end{column}
    \begin{column}{0.45\textwidth}
      \begin{figure}
        \includegraphics[width=1.7in]{tiling_tiled_step_3.png}
      \end{figure}
    \end{column}
  \end{columns}
}

\frame{
  \frametitle{Representing Collision Strings}

  \begin{example}
    Tiling of $\vec{x}_0 = (0, 0.5)$ and $\vec{v} = (0.25, 0.25)$.
  \end{example}

  \begin{columns}
    \begin{column}{0.45\textwidth}
      \begin{figure}
        \includegraphics[width=2.1in]{tiling_real_step_4.png}
      \end{figure}
    \end{column}
    \begin{column}{0.45\textwidth}
      \begin{figure}
        \includegraphics[width=1.7in]{tiling_tiled_step_4.png}
      \end{figure}
    \end{column}
  \end{columns}
}

\frame{
  \frametitle{Representing Collision Strings}

  \begin{example}
    Tiling of $\vec{x}_0 = (0, 0.5)$ and $\vec{v} = (0.25, 0.25)$.
  \end{example}

  \begin{columns}
    \begin{column}{0.45\textwidth}
      \begin{figure}
        \includegraphics[width=2.1in]{tiling_real_step_1.png}
      \end{figure}
    \end{column}
    \begin{column}{0.45\textwidth}
      \begin{figure}
        \includegraphics[width=1.7in]{tiling_tiled_step_5.png}
      \end{figure}
    \end{column}
  \end{columns}
}

\frame{
  \frametitle{Representing Collision Strings}

  \begin{example}
    Tiling of $\vec{x}_0 = (0, 0.5)$ and $\vec{v} = (0.25, 0.25)$.
  \end{example}

  \begin{columns}
    \begin{column}{0.45\textwidth}
      \begin{figure}
        \includegraphics[width=2.1in]{tiling_real_step_2.png}
      \end{figure}
    \end{column}
    \begin{column}{0.45\textwidth}
      \begin{figure}
        \includegraphics[width=1.7in]{tiling_tiled_step_6.png}
      \end{figure}
    \end{column}
  \end{columns}
}

\frame{
  \frametitle{Representing Collision Strings}

  \begin{example}
    Tiling of $\vec{x}_0 = (0, 0.5)$ and $\vec{v} = (0.25, 0.25)$.
  \end{example}

  \begin{columns}
    \begin{column}{0.45\textwidth}
      \begin{figure}
        \includegraphics[width=2.1in]{tiling_real_step_3.png}
      \end{figure}
    \end{column}
    \begin{column}{0.45\textwidth}
      \begin{figure}
        \includegraphics[width=1.7in]{tiling_tiled_step_7.png}
      \end{figure}
    \end{column}
  \end{columns}
}

\frame{
  \frametitle{Periodicity of Rationals}

  We can represent a particle as a line in the plane $y = mx + y_0$ in the tiling.

  Periodicity occurs if $y_0 2k = mx + y_0$ for some $k \in \mathrm{N}$.

  \begin{itemize}
    \item If $m, y_0 \in \mathrm{Q}$, then all sequences are periodic.
    \item If $m$ or $y_0$ are irrational, then the sequences are not periodic.
  \end{itemize}
}

\frame{
  \frametitle{Maximum Differences between Primary Substring Lengths}

  Given a collision string, how different can primary substrings be?

  \begin{example}
    Is $abaaab$ possible?
  \end{example}
}

\frame{
  \frametitle{Maximum Differences between Primary Substring Lengths}

  Length of primary substring $i$ is given by:
  \begin{eqnarray}
    L(i, m, y_0) = \floor{\frac{i - y_0}{m}} - \floor{\frac{i-1 - y_0}{m}}
  \end{eqnarray}
}

\frame{
  \frametitle{Maximum Differences between Primary Substring Lengths}

  For $i, j \in \mathrm{N}$, $m \in [0,1]$, $y_0 \in [0,1]$, we will show:

  \begin{eqnarray}
    \max_{i > j} L(i, m, y_0) - L(j, m, y_0) \leq 2
  \end{eqnarray}
}

\frame{
  \frametitle{Maximum Differences between Primary Substring Lengths}

  \begin{lemma}
    \begin{eqnarray}
      \floor{a - x} + \floor{b - x} = \floor{a} - \floor{b} + (\floor{\fracp{a} - \fracp{x}} - \floor{\fracp{b} - \fracp{x}})
    \end{eqnarray}
  \end{lemma}

  \begin{eqnarray}
  L(i, m, y_0) - L(j, m, y_0) &=& \left( \floor{\frac{i}{m}} - \floor{\frac{i-1}{m}} \right) + \left( \floor{\frac{j}{m}} - \floor{\frac{j-1}{m}} \right) \\
                              &+& \left( \floor{\fracp{\frac{i}{m}} - \fracp{\frac{y_0}{m}}} - \floor{\fracp{\frac{i-1}{m}} - \fracp{\frac{y_0}{m}}} \right) \\
                              &-& \left( \floor{\fracp{\frac{j}{m}} - \fracp{\frac{y_0}{m}}} - \floor{\fracp{\frac{j-1}{m}} - \fracp{\frac{y_0}{m}}} \right) 
  \end{eqnarray}
}

\frame{
  \frametitle{Maximum Differences between Primary Substring Lengths}

  Let $g(x, m, y_0) = \floor{\fracp{\frac{x}{m}} - \fracp{\frac{y_0}{m}}} - \floor{\fracp{\frac{x-1}{m}} - \fracp{\frac{y_0}{m}}}$.

  \begin{eqnarray}
  L(i,m,y_0) - L(j,m,y_0) &=& \fracp{\frac{i-1}{m}} - \fracp{\frac{i}{m}} - \left(\fracp{\frac{j-1}{m}} - \fracp{\frac{j}{m}} \right) \\
                          &+& g(i,m,y_0) - g(j,m,y_0)
  \end{eqnarray}
}

\frame{
  \frametitle{Maximum Differences between Primary Substring Lengths}
  \begin{eqnarray}
  L(i,m,y_0) - L(j,m,y_0) &=& \fracp{\frac{i-1}{m}} - \fracp{\frac{i}{m}} - \left(\fracp{\frac{j-1}{m}} - \fracp{\frac{j}{m}} \right) \\
                          &+& g(i,m,y_0) - g(j,m,y_0)
  \end{eqnarray}

  \begin{itemize}
    \item If $g(x,m,y_0) = 1$, then:
      \begin{eqnarray}
        \fracp{\frac{x-1}{m}} < \fracp{\frac{y_0}{m}} < \fracp{\frac{x}{m}}
      \end{eqnarray}

      Which implies $0 < L(x,m,y_0) < 1$.

    \item If $g(x,m,y_0) = -1$, then:
      \begin{eqnarray}
        \fracp{\frac{x}{m}} < \fracp{\frac{y_0}{m}} < \fracp{\frac{x-1}{m}}
      \end{eqnarray}

      Which implies $-1 < L(x,m,y_0) < 0$.

    \item If $g(x,m,y_0) = 0$, then:
      $-1 < L(x,m, y_0) < 1$.
  \end{itemize}
}

\frame{
  \frametitle{Maximum Differences between Primary Substring Lengths}

  \begin{eqnarray}
    \max_{i > j} L(i,m,y_0) - L(j,m,y_0) \leq 2
  \end{eqnarray}
}

\subsection{1-dimensional Problem}

\frame{
  \frametitle{Sequence Characterization}

  \begin{multline*}
    \begin{array}{cccccccccc}
      \underbrace{aaa} & b & \underbrace{aa} & b & \underbrace{aa} & b & \underbrace{aaa} & b & \underbrace{aa} & b \\
      3 && 2 && 2 && 3 && 2 &
    \end{array}\\
    \begin{array}{cccccccc}
      \underbrace{aa} & b & \underbrace{aaa} & b & \underbrace{aa} & b & \underbrace{aaa} & \dots \\
      2 && 3 && 2 && 3 & \dots
    \end{array}
  \end{multline*}
}

\frame{
  \frametitle{Sequence Characterization}

  \begin{gather*}
    \begin{array}{cccccccc}
      3 & \underbrace{22} & 3 & \underbrace{22} & 3 & \underbrace{2} & 3 & \dots\\ 
      & 2 && 2 && 1 & \dots
    \end{array}
  \end{gather*}
}


\section{Algorithm}

\frame{

\frametitle{Trajectory Calculation Algorithm}
	
\begin{enumerate}
\item Let $i = 0$
\item 
\end{enumerate}

}

\section{Future Research}

\subsection{Tileable Polygons}

\frame{
  \frametitle{Extensions to Tileable Polygons}

  Other Tileable Polygons:

  \begin{columns}
    \begin{column}{0.45\textwidth}
      \begin{figure}
        \includegraphics[height=1.5in]{hexagon_tiling.png}
        \caption{Regular Hexagons}
      \end{figure}
    \end{column}
    \begin{column}{0.45\textwidth}
      \begin{figure}
        \includegraphics[height=1.5in]{triangle_tiling.png}
        \caption{Equilateral Triangles}
      \end{figure}
    \end{column}
  \end{columns}
}

\subsection{Non-Tileable Polygons}

\frame{
  \frametitle{Extensions to Non-Tileable Polygons}
  \begin{itemize}
    \item Irregular triangles
    \item Pentagons
    \item Octagons
  \end{itemize}
}

\subsection{Circles}

\frame{
  \frametitle{Extensions to Circles}

  \begin{itemize}
    \item 
  \end{itemize}
}




\frame{
This is a slide!
}


\frame{
\frametitle{This slide has a title!}

\pause
Wow!
\pause

You can also make lists...
\begin{itemize}
	\item with an item \pause
	\item and another item.
\end{itemize}

}


\frame{
You can typeset equations just as in latex:
\[
\int_0^1 x\, dx = 1/2,
\]
or
$$
\sum_{i = 1}^\infty i = -\frac{1}{12}
$$
or
\begin{equation*}
1 - 1 + 1 - \cdots = \frac{1}{2}.
\end{equation*}

}


\frame{
If you want a number for an equation, do it like this:
\begin{equation}\label{eq:first-equation}
\lim_{n \to \infty}\, \sum_{k = 1}^n \frac{1}{k^2} = \frac{\pi}{6}.
\end{equation} %
This can then be referred to as \eqref{eq:first-equation}, which is much easier than
keeping track of numbers by hand. To group several equations, aligning on the $=$ sign, do
it like this:
\begin{align*}
x_1 + 2x_2 + 3x_3 &= 7 \\ y &= mx + c \\ &= 4x - 9.
\end{align*}
}

\frame{
\begin{theorem}[my great theorem]
	This theorem proves everything.
\end{theorem}

\begin{example}
	For example it proves \alert{this}! % "alert" makes it red
\end{example}

}



\end{document}

