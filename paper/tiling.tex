We shall now present a representation of the problem which will greatly simplify the analysis of $v$ and $h$ collisions for some ball $b$, called the tiling representation.

To understand the basics of how it works, imagine placing a square billiard table on the $xy$ plane. The billiard table (as mentioned in the introduction) will be a unit square, so it will contain $[0,1]^2$. The table's edges will be the four line segments bordering the unit square. A ball will start with some initial position $\bvec{x}_0 \in [0,1]^2$ and velocity $\bvec{v}_0$. After some time, the ball will collide with an edge $e_0$ of the table. However, instead of thinking of the trajectory of the ball as being reflected across the line perpendicular to $e$ at the point of collision, we will instead reflect the original unit square $s_0$ across the edge $e_0$ to create a new square $s_1$. Now, the trajectory of the ball after the first collision will be presented in the new square $s_1$.

In other words, the trajectory of $b$ before the first collision will be confined to the original square $s_0$, and the trajectory after the first collision will be confined to the new reflected square $s_1$. We can continue the process for each new collision. Suppose the ball collides with edge $e_1$ in square $s_1$. Then, we shall create a new square $s_2$ which is a reflection of square $s_1$ across the edge $e_1$. The trajectory of the ball $b$ after the second collision will be confined to the newest reflected square $s_2$. This process will continue on indefinitely.

