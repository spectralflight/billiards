Finally, we will show how collision sequences relate to continued fractions. A simple continued fraction of a real number $r$ (hereafter referred to as a continued fraction) is the expression given below in equation \ref{eq:continued-fraction}:

\begin{equation}
  \label{eq:continued-fraction}
  r = k_1 + \frac{1}{k_2 + \frac{1}{k_3 + \cdots}}
\end{equation}

Where $k_1 = \floor{r}, r_1 = \frac{1}{r-k_1}, k_2=\floor{r_1}, r_2 = \frac{1}{r_1-k_2}, \ldots$. This process terminates when $r_i$ becomes an integer. The integers $k_i$ are called ``partial quotients'' and a real number $r$ can be expressed in its continued fraction form $r = [k_1, k_2, k_3, \ldots]$. As an example, we can find the continued fraction for $r = 3.245$.

\begin{example}
  \begin{equation}
    3.245 = 3 + \frac{1}{4 + \frac{1}{12 + \frac{1}{4}}}
  \end{equation}

In this example, we see that $k_1 = \floor{r} = \floor{3.245} = 3$ by the definition. Thus, we can define $r_1 = \frac{1}{3.245 - 3} = \frac{1}{0.245} = \frac{200}{49}$ and find $k_2 = \floor{r_1} = \floor{\frac{200}{49}} = 4$. Now we can recursively define $r_2 = \frac{1}{\frac{200}{49} - 4} = \frac{49}{4}$. As we continue onwards in this process, we find $k_3 = 12$, $r_3 = 4$, and $k_4 = 4$. Thus, the continued fraction representation is $3.245 = [3,4,12,4]$.
\end{example}

If we use the continued fraction representation for the slope $m$ of a combined trajectory for a billiard ball, then we have:

\begin{equation}
  m = \floor{m} + \frac{1}{\floor{1 / \cbracket{m}} + \frac{1}{\floor{1/\cbracket{1/\cbracket{m}}} + \cdots}}
\end{equation}

Where the partial quotients are given by:
\begin{align*}
  k_1 &= \floor{m} \\
  k_2 &= \floor{\frac{1}{\cbracket{m}}} \\
  k_3 &= \floor{\frac{1}{\cbracket{ \frac{1}{\cbracket{m}}}}} \\
  k_4 &= \floor{\frac{1}{\cbracket{ \frac{1}{\cbracket{\frac{1}{\cbracket{m}}}}}}} \\
      &\vdots
\end{align*}

We see that the partial quotients are given by the recursive formula: $k_j = \floor{\frac{1}{m - [k_1, k_2, \ldots, k_{j-1}]}}$. In fact, a more interesting observation is that the sequence of partial quotients $k_j$ form exactly the sequence of the minimum number of tick marks in each $\beta^{(j)}$ subproblem:

\begin{theorem}
  Given the continued fraction representation $m = [k_1, k_2, k_3, \ldots]$ for the slope $m \in \mathrm{R}$ of the combined trajectory $T$ of a billiard ball, we must have $k_j = \beta_{min}^{(j)}$.
\end{theorem}
\begin{proof}
  We will proceed by induction. This is clearly true for $k_1 = \floor{m}$ by lemma \ref{lem:interval-ticks}, since for $\beta_{min}^{(j)}$ the spacing of the larger tick marks are $m$ and the spacing of the smaller tick marks is $1$.

  Now suppose we have shown our hypothesis to be true for all $k_1, k_2, \ldots, k_{j-1}$. This means that $\beta_{min}^{(i)} = k_{i}$ for all $i \leq j-1$. By definition \ref{def:beta-definition} (of the sequence $\beta^{(j)}$), we see that $\beta_i^{(j)}$ is one more than the number of occurrences of $\beta^{(j-1)}_{max}$ between the $i$th and $(i+1)$st occurrence of $\beta^{(j-1)}_{min}$ in the $\beta^{(j-1)}$ sequence.

  Now, we can find an expression for $k_{j}$ based on our definition of continued fractions:
  \begin{align}
    k_j &= \floor{\frac{1}{m - [k_1, k_2, \ldots, k_{j-1}]}} \\
        &= \floor{\frac{1}{m - [\beta_{min}^{(1)}, \beta_{min}^{(2)}, \ldots, \beta_{min}^{(j-1)}]}}
  \end{align}

  However, notice that $d = m - [\beta_{min}^{(1)}, \ldots, \beta_{min}^{(j-1)}]$ is equal to the spacing between tick marks in the $\beta^{(j)}$ sequence. Thus, we see that $\beta_{min}^{(j)} = \floor{1/d}$. However, this is exactly what we wanted to show, since we know that $k_j = \floor{1/d}$, so we see that $\beta_{min}^{(j)} = k_j$.
\end{proof}

In fact, one can see that the process of finding $\beta^{(j)}_i$ for the $\beta^{(j)}$ subproblems exactly mirrors the process of finding the partial quotients $k_j$ in a continued fraction of $m$.
