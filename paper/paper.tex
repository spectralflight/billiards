\documentclass[12pt]{amsart}   % LaTeX with AMS style; 12 point for old eyes

\usepackage{paper}

\begin{document}

\graphicspath{ {figures/} }

\title[Billiards]{Sequences of Billiard Ball Collisions}

\author{Jonathan Allen, John Wang}
\date{\today}

\maketitle

\subsection*{Abstract}

In this paper we explore properties of sequences of billiard ball collisions. We present a tiling representation which is used to help elucidate some simple properties of these sequences. Then, we provide a one-dimensional representation which builds upon the tiling representation. Next, we provide a method which takes an arbitrary sequence and checks to see if it could have been created by a billiard ball colliding with a billiard table. Finally, we show how these sequences relate to continued fractions.

\section{Introduction}
\frame{
  \frametitle{Introduction}

  \begin{itemize}
    \item Billiard ball bouncing in a square
    \item Assume no gravity or friction
  \end{itemize}
}

\subsection{Basic Notation}

\frame{
  \frametitle{Basic Notation}
  \begin{definition}
    A table $T \subset \R^2$ is the unit square. Vertical sides are labelled with a $v$. Horizontal sides are labelled with an $h$.
  \end{definition}

  \begin{figure}
    \includegraphics[width=2in]{square_with_sides.png}
  \end{figure}
}

\subsection{Example}

\frame{
  \frametitle{Example}

  \begin{eqnarray*}
    {\color{red}vv}
  \end{eqnarray*}
  \begin{figure}
    \includegraphics[width=3in]{example/example_1.png}
  \end{figure}
}

\frame{
  \frametitle{Example}

  \begin{eqnarray*}
    vv{\color{red}vhv}
  \end{eqnarray*}
  \begin{figure}
    \includegraphics[width=3in]{example/example_2.png}
  \end{figure}
}
\frame{
  \frametitle{Example}

  \begin{eqnarray*}
    vvvhv{\color{red}vvv}
  \end{eqnarray*}
  \begin{figure}
    \includegraphics[width=3in]{example/example_3.png}
  \end{figure}
}

\frame{
  \frametitle{Example}

  \begin{eqnarray*}
    vvvhvvvv{\color{red}vhv}
  \end{eqnarray*}
  \begin{figure}
    \includegraphics[width=3in]{example/example_4.png}
  \end{figure}
}

\frame{
  \frametitle{Example}

  \begin{eqnarray*}
    vvvhvvvvvhv{\color{red}vv}
  \end{eqnarray*}
  \begin{figure}
    \includegraphics[width=3in]{example/example_5.png}
  \end{figure}
}

\frame{
  \frametitle{Example}

  \begin{eqnarray*}
    vvvhvvvvvhvvv{\color{red}vhv}
  \end{eqnarray*}
  \begin{figure}
    \includegraphics[width=3in]{example/example_6.png}
  \end{figure}
}

\frame{
  \frametitle{Resulting Sequence}
  \begin{eqnarray*}
    vvvhvvvvvhvvvvhv
  \end{eqnarray*}
}

\subsection{Outline}

\frame{
  \frametitle{Presentation Outline}
  \tableofcontents
}

\subsection{Problem Statement}

\frame{
  \frametitle{Problem Statement}

   Problem: Given a sequence of $v$ and $h$ collisions, determine if it is a valid collision sequence.
}

\frame{
  \frametitle{Basic Notation}

  \begin{definition}
    \emph{$v$ collision}: when the ball collides with a $v$ side
  \end{definition}

  \begin{definition}
    \emph{$h$ collision}: when the ball collides with an $h$ side
  \end{definition}

  \begin{definition}
    \emph{Collision sequence ($\alpha$)}: a sequence of $v$ and $h$ collisions which starts and ends with an $h$ collision.
  \end{definition}
}


\section{Tiling Representation}
\frame{
  \frametitle{Representing Collision Sequences}

  \begin{itemize}
    \item Tile the table in the plane for a more powerful representation of the problem
    \item Tiling will reflect the table about each side
  \end{itemize}
}

\frame{
  \frametitle{Tiling Tables}
  \begin{figure}

  \end{figure}
}


\section{Simple Properties}
We will now use the tiling representation that we have developed to discover properties of collision sequences. The first simple property is that collision sequences are periodic when the initial conditions are rational numbers. Now that we can define a billiard ball as a line, instead of giving initial conditions $\bvec{x}_0$ and $\bvec{v}_0$, we can give the slope $m$ and the $y$-intercept $y_0$ of the complete trajectory's line. The formalized theorem is then:

\begin{theorem}
  \label{theorem:periodicity}
  There exists a $k \in \mathrm{N}$ such that $mk + y_0 \equiv y_0 \pmod{2}$ if and only if $m \in \mathrm{Q}$.
\end{theorem}
\begin{proof}
\end{proof}

Theorem \ref{theorem:periodicity} shows that if $m \in \mathrm{Q}$, then the billiard ball will eventually return to it's original position $\bvec{x}_0$ with its original velocity $\bvec{v}_0$. Seeing why this is true is just a matter of using the tiling representation. We note that every second square in either the $x$ or $y$ direction is the same (because of the transitivity of reflection). Therefore, every second square will have a trajectory that exactly corresponds to the trajectory in the original square.


\section{1-Dimensional Representation}
%!TEX root = paper.tex

Rather than looking at an explicit representation of lines in the plane, we can gain much more insight from looking at a parametric representation. To simplify our analysis, we will choose our time parameter such that v collisions occur every $\Delta t = 1$ and h collisions occur every $\Delta t = m$. The equation for a line $y(x) = m \, x + b$ is equivalent to the following parametric system

\begin{align}\label{eq:parametric-line}
	x(t) = t + x_0\\
	y(t) = m \, t
\end{align}

Now our v and h collisions in the 2-dimensional plane can be projected onto the 1-dimensional parametric representation.

\begin{figure}[H]
  \begin{center}
    \includegraphics[keepaspectratio, width=4in]{1d_mapping_2.png}
  \end{center}
  \vspace{-.2in} % corrects bad spacing
  \caption{\label{fig:1d-projection} Projecting onto the parametric representation.}
\end{figure}

We will occasionally have to deal with some boundary conditions, and we will introduce some intermediate sequences which we will label as `augmented' and mark with a tilde. The boundary treatments are fairly pedantic and can be ignored if the reader is just interested in getting a general understanding of our solutions.

All of our theorems in this section will rely on the lengths of various patterns in the original collision sequence. We will start off by looking at the lengths of subsequences of v collisions in the collision sequence. To make this accounting simpler, we need to define an augmented collision sequence.

\begin{definition}
	\textbf{augmented collision sequence ($\tilde{\alpha}$)} which consists of the original collision sequence with one h collision added to both ends of the sequence.
\end{definition}

Now we can count lengths of v collision substrings

\begin{definition}
	\textbf{Augmented $\beta_i$:} number of v collisions between i\textsuperscript{th} and (i+1)\textsuperscript{th} h collisions in the augmented collision sequence
\end{definition}

\textbf{Still working on this part, not sure how best to deal with the boundary conditions...}

The first and last numbers in the $\beta$ sequence were artificially created by augmenting our original collision sequence. These two numbers only give us a lower bound on the number of v collisions between h collisions, so we can safely discard them if 

\begin{definition}
	\begin{align}
		\beta_{min} \coloneqq \min_i \beta_i\\
		\beta_{max} \coloneqq \max_i \beta_i
	\end{align}
\end{definition}

The $\beta$ sequence is much simpler to think of geometrically in terms of our parametric representation shown in Figure \ref{fig:1d-projection}. $\beta_i$ represents the number of v collision tick marks in between each h collision tick mark.

\begin{figure}[H]
  \begin{center}
    \includegraphics[keepaspectratio, width=4in]{1d_mapping_3.png}
  \end{center}
  \vspace{-.2in} % corrects bad spacing
  \caption{\label{fig:beta-sequence} The $\beta$ sequence.}
\end{figure}

\begin{lemma}\label{lemma:beta_pos}
	For every valid collision sequence, the following must be true

	\begin{equation}
		\beta_{min} > 0
	\end{equation}
\end{lemma}

\begin{proof}
	TODO
\end{proof}

\begin{theorem}\label{thm:beta_exremum}
	For every valid collision sequence, the following must be true
	
	\begin{equation}
		\beta_{max} - \beta_{min} \le 1
	\end{equation}
\end{theorem}

\begin{proof}

From Equation \ref{eq:parametric-line}, v collisions occur every $\Delta t = 1$ and h collisions occur every $\Delta t = m$. Thus, the following must be true

\begin{equation}
	\beta_i \in \paren{\floor{m}, \ceil{m}}
\end{equation}

For an $m$ to exist that satisfies the above constraints, all numbers in the $\beta$ sequence can only differ by 1.

\end{proof}

\begin{definition}
	\textbf{Augmented $C^{(0)}_i$:} 1 more than the number of occurrences of $\beta_{max}$ between i\textsuperscript{th} and (i+1)\textsuperscript{th} occurrence of $\beta_{min}$ in the $\beta$ sequence.
\end{definition}

\begin{theorem}\label{thm:beta_i}
	Define 
	\begin{align}\label{delta_beta}
			\delta^{(\beta)}_i \coloneqq \begin{cases}
				x_0 \qquad &\text{if} \quad i = 0\\
				i (\beta_{max} - m) \qquad &\text{otherwise}
			\end{cases}
	\end{align}

	Then the following is true for all valid collision sequences

	\begin{align}\label{eq:beta_i}
		\beta_i = \floor{\delta^{(\beta)}_i} + \beta_{max} - \floor{\delta^{(\beta)}_{i+1}}
	\end{align}
\end{theorem}

\begin{proof}
	TODO...
\end{proof}

We can immediately notice that the $\delta^{(\beta)}$ sequence has the following features:

\begin{enumerate}
	\item The $\delta^{(\beta)}$ sequence is increasing, because $\beta_{max} \ge m$
	\item Combining Theorem \ref{thm:beta_exremum} and Equation \ref{eq:beta_i}, we get the following:

		\begin{align}
			\floor{\delta^{(\beta)}_{i+1}} - \floor{\delta^{(\beta)}_{i+1}}& = \beta_{max} - \beta_{min}\\
			& \le 1
		\end{align}
\end{enumerate}

Thus, if we plot the values of the $\delta^{(\beta)}$ sequence on a line, we notice something interesting: the plot looks very similar to our original plot of the collision sequence parameterized by $t$. 

\begin{theorem}
	Define 
	\begin{align}\label{delta_c}
			\delta^{(C^{(j)})}_i \coloneqq \begin{cases}
				x_0 \qquad &\text{if} \quad i = 0\\
				i (C^{(j)}_{max} - m) \qquad &\text{otherwise}
			\end{cases}
	\end{align}

	Then the following is true for all valid collision sequences

	\begin{align}\label{eq:c_i}
		C^{(j)}_i = \floor{\delta^{(C^{(j)})}_i} + C^{(j)}_{max} - \floor{\delta^{(C^{(j)})}_{i+1}}
	\end{align}
\end{theorem} 

\begin{theorem}
	Define the sequence $a$ as

	\begin{align}
		a_0& \coloneqq 1\\
		a_1& \coloneqq \beta_{max} - m\\
		a_i& \coloneqq C^{(i-2)} a_{i-1} - a_{i-2} \qquad \text{for} \quad i \ge 2
	\end{align}

	For every valid collision sequence, $a \to 0$
\end{theorem}

\begin{proof}

\end{proof}

\section{Some Useful Sequences}
%!TEX root = paper.tex

To analyze the 1-dimensional problem, we will derive a group of sequences from the original collision sequence, which we will call the $\beta$ group. This group will allow us to validate any collision sequence. The first sequence in this group is defined below.

\begin{definition}
	Given a collision sequence $\alpha$, define a sequence $\beta^{(0)}$ where each element $\beta^{(0)}_i$ is the number of h's between the $i^{th}$ and $(i+1)^{th}$ v in $\alpha$. From Lemma \ref{lem:interval-ticks}, each element in $\beta^{(0)}$ can be one of two different values, which we will refer to as $\beta^{(0)}_{min}, \beta^{(0)}_{max}$.
\end{definition}

Graphically, $\beta^{(0)}_i$ represents the number of tick marks in each interval. An example sequence is shown in Figure \ref{fig:beta-sequence}.

\begin{figure}[H]
  \begin{center}
    \includegraphics[keepaspectratio]{1d_mapping_3}
  \end{center}
  \vspace{-.2in} % corrects bad spacing
  \caption{\label{fig:beta-sequence} An example $\beta^{(0)}$ sequence.}
\end{figure}

 We will also need a new sequence $\delta^{(0)}$ where each element is spaced $-1 \, \cbracket{\frac{m}{1}}$ apart. $\delta^{(0)}$ will also include an offset of $(1-y_0)$ for reasons that will become clear shortly. In terms of our $\beta$ group, $\cbracket{\frac{m}{1}}$ can be written as 

\begin{equation}
  \cbracket{\frac{m}{1}} = \frac{m}{1} - \beta^{(0)}_{min}
\end{equation}

So, we can write $\delta^{(0)}$ in terms of $\beta$

\begin{definition}
  $\delta^{(0)}$ is defined more precisely as

  \begin{align}\label{delta_beta}
    \delta^{(0)}_i \coloneqq \begin{cases}
      1-y_0 \qquad &\text{for} \quad i = 0\\
      i (\beta^{(0)}_{min} - m) + (1-y_0) \qquad &\text{for} \quad i \ge 1
    \end{cases}
  \end{align}
\end{definition}

Visually, each element $\delta^{(0)}_i$ is the signed distance between the beginning of the $i^{th}$ interval and the $(i \, \beta^{(0)}_{min})^{th}$ tick mark. Figure \ref{fig:delta-sequence} shows the $\delta^{(0)}_i$ sequence on top of the original parametric representation (top) as well as by itself (bottom).

\begin{figure}[H]
  \begin{center}
    \includegraphics[keepaspectratio]{1d_mapping_4}
  \end{center}
  \vspace{-.2in} % corrects bad spacing
  \caption{\label{fig:delta-sequence} Generating the $\delta^{(0)}$ sequence.}
\end{figure}

Our next step is to define $\beta^{(j)}, \delta^{(j)}$ for $j \ge 1$, which will be done in a inductive manner. Before continuing, lets create a third sequence, that we will call $a$.

\begin{definition}
  \begin{equation}
    a_j \coloneqq \begin{cases}
      m \qquad &\text{for} \quad i = -2\\
      1 \qquad &\text{for} \quad i = -1\\
      a_{j-2} - \beta^{(j)} a_{j-1} &\qquad \text{for} \quad i \ge 0
    \end{cases}
  \end{equation}
\end{definition}

$a_{-2}, a_{-1}$ are the interval size and tick spacing in our original problem. $a_0$ is the spacing of elements of $\delta^{(0)}$. $a_j$ for $j \ge 1$ will represent the spacing of elements of $\delta^{(j)}$ sequences which will be defined shortly. For the moment, the reader should assume that, given some $j$, $\delta^{(j-1)}$ exists, and l

\begin{figure}[H]
  \begin{center}
    \includegraphics[keepaspectratio]{1d_mapping_6}
  \end{center}
  \vspace{-.2in} % corrects bad spacing
  \caption{\label{fig:delta-sequence-2} An example $\delta^{(0)}$ sequence.}
\end{figure}

We will now build off of the $\beta^{(0)}$ sequence, and extend the $\beta$ group inductively.

\begin{definition}
  \label{def:beta-definition}
  Given a $j > 0$ and a collision sequence $\alpha$, assume that $\beta^{(j-1)}$ is defined and each element in the sequence is either $\beta^{(j-1)}_{min}$ or $\beta^{(j-1)}_{max}$. The sequence $\beta^{(j)}$ is defined such that each element $\beta^{(j)}_i$ is 1 more than the number of occurrences of $\beta^{(j-1)}_{max}$ between the $i^{th}$ and $(i+1)^{th}$ occurrence of $\beta^{(j-1)}_{min}$ in the $\beta^{(j-1)}$ sequence. From Lemma \ref{lem:interval-ticks}, each element in $\beta^{(j)}$ can be one of two different values, which we will refer to as $\beta^{(j)}_{min}, \beta^{(j)}_{max}$.

  If, for some $j_f$, the length of $\beta^{(j_f-1)}$ is 1, then $\beta^{(j_f-1)}$ is called the terminating $\beta$ sequence, and all subsequent $\beta^{(j)}$ for $j \ge j_f$ are undefined.
\end{definition}

$\beta^{(j)}$ is much simpler to understand visually. A visualization representation is formed using the following rules:

\begin{enumerate}
  \item Divide the number line into regular intervals of length $a_{j}$.
  \item $\beta^{(j-1)}_{min}$ are represented as tick marks on the number line with a regular spacing $a_{j+1}$.
  \item There is exactly one $\beta^{(j-1)}_{min}$ tick mark in each interval.
  \item In each interval, all the $\beta^{(j-1)}_{max}$ terms come before the $\beta^{(j-1)}_{min}$ term.
\end{enumerate}

Then $\beta^{(j)}_i$ is the number of tick marks in each interval. An example $\beta^{(j-1)}$ visual representation is shown in Figure \ref{fig:beta-sequence-j}.

\begin{figure}[H]
  \begin{center}
    \includegraphics[keepaspectratio]{1d_mapping_7}
  \end{center}
  \vspace{-.2in} % corrects bad spacing
  \caption{\label{fig:beta-sequence-j} A general $\beta^{(j)}$ sequence.}
\end{figure}


Lastly, we need to extend our $\delta$ group, such that each $\delta^{(j)}_i$ is the distance between the $(i \, \beta^{(j)}_{max})^{th}$ tick mark and the beginning of the $i^{th}$ interval in the $\beta^{(j)}$ visual representation.

\begin{definition}
  $\delta^{(j)}$ is defined more precisely as

  \begin{align}\label{delta_beta}
    \delta^{(j)}_i \coloneqq \begin{cases}
      1-y_0 \qquad &\text{for} \quad i = 0\\
      i (\beta^{(j)}_{min} * a_{j-1} - a_{j-2}) + (1-y_0) \qquad &\text{for} \quad i \ge 1
    \end{cases}
  \end{align}
\end{definition}

The number of tick marks in the $i^{th}$ interval is thus equal to $\beta^{(j)}_{max}$ plus the number of tick marks included in the $\delta^{(j)}_i$ interval minus the number of tick marks included in the $\delta^{(j)}_{i+1}$ interval. More precisely

\begin{align}\label{eq:beta_i}
  \beta^{(j)}_i = \floor{\delta^{(j)}_i} + \beta^{(j)}_{max} - \floor{\delta^{(j)}_{i+1}}
\end{align}


\section{Satisfiability Conditions}
%!TEX root = paper.tex

\begin{lemma}\label{lem:h-extension}
	A finite sequence $\alpha$ is a valid collision sequence iff there exists at least one valid collision sequence containing $\alpha$ that starts and ends with an h.
\end{lemma}

\begin{proof}
	TODO: can extend any collision sequence to any length. 
\end{proof}

Because of Lemma \ref{lem:h-extension}, without loss of generality we can confine ourselves to only look at collision sequences that start and end with an h.

% -----------------------------------------------------------------------------

We will now show how the sequences in the $\beta$ group relate to the original problem of validating a collision sequence. In the process we will generate the sequence of interval sizes $a$. 

\textbf{TODO: still trying to figure out how to explain how the $\beta$ sequences relate to the original problem}

We start with $a_0 \coloneqq m, a_1 \coloneqq 1$, which are the spacing of v collisions and h collisions respectively in the parametric representation. The rest of the $a$ sequence is generated inductively. 

For some $2 \le j < j_f$, assume that $a_{j-1}, a_{j-2}$ exist.

\begin{align}
	a_{j} = \beta^{(j-2)}_{max} a_{j-1} - a_{j-2} \qquad &\text{for} \quad 2 \le j < j_f
\end{align}

From now on we will only consider collision sequences, where each non-terminal $\beta^{(j)}$ starts and ends with $\beta^{(j)}_{min}$.

% -----------------------------------------------------------------------------

\begin{theorem}\label{thm:beta_i}
	A collision sequence is valid if the following is true for all $j$

	\begin{equation}
		\beta^{(j)}_i \in \cbracket{\floor{\frac{a_{j-2}}{a_{j-1}}}, \ceil{\frac{a_{j-2}}{a_{j-1}}}}
	\end{equation}
\end{theorem}

\begin{proof}
	This follows directly from Lemma \ref{lem:interval-ticks} and the definition of $\beta^{(j)}$.
\end{proof}

% -----------------------------------------------------------------------------

\begin{theorem}
	For every valid collision sequence, ${\displaystyle \lim_{n \to \infty} a_n = 0}$
\end{theorem}

\begin{proof}
	From the definition of $a_j$

	\begin{align}\label{eq:a-def-2}
		a_j& = \beta^{(j-2)}_{max} \, a_{j-1} - a_{j-2}\\
		& =  \ceil{\frac{a_{j-2}}{a_{j-1}}} a_{j-1} - a_{j-2}
	\end{align}

	Now from the definition of the ceiling function, we know that

	\begin{equation}
		0 \le \ceil{\frac{a_{j-2}}{a_{j-1}}} - \frac{a_{j-2}}{a_{j-1}} < 1
	\end{equation}

	Rearranging, we get

	\begin{equation}
		0 \le \ceil{\frac{a_{j-2}}{a_{j-1}}} a_{j-1} - a_{j-2} < a_{j-1}
	\end{equation}

	Combining the above equation with Equation \ref{eq:a-def-2}, we get

	\begin{equation}
		0 \le a_j < a_{j-1}
	\end{equation}

	Thus $a$ is strictly decreasing and bounded below by 0, so ${\displaystyle \lim_{n \to \infty} a_n = 0}$.
\end{proof}


\section{Continued Fractions}
Finally, we will show how collision sequences relate to continued fractions.


\section{Conclusion}
%!TEX root = paper.tex

In this paper we explored properties of valid billiard ball collision sequences. We introduced a useful tiling representation that drastically simplified the problem and allowed us to easily characterize all valid collision sequences. From the tiling representation, we developed a series of ``meta'' problems in the same manner as one calculates a continued fraction expansion. We also showed an interesting way to visualize the original problem and the ``meta'' problems as 1-dimensional problems on a number line.


%!TEX root = paper.tex

\bibliographystyle{plain}
\begin{thebibliography}{9}
\bibitem{wikipedia_cfrac}
	Continued Fraction. (2013, 11 30). Retrieved from Wikipedia: \url{http://en.wikipedia.org/wiki/Continued_fraction}
\end{thebibliography}

\end{document}
